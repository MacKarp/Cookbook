%ustawienia
\documentclass[12pt,a4paper]{article}
\usepackage[T1]{fontenc}
\usepackage{mathptmx}
\usepackage[utf8]{inputenc}
\usepackage{amssymb}
\usepackage[polish]{babel}
\usepackage{polski}
\usepackage{amsmath}
\usepackage{amsfonts}
\usepackage[left=3.5cm,right=2cm,top=2.5cm,bottom=2.5cm]{geometry}
\usepackage{graphicx}
\usepackage{indentfirst} 
\usepackage{float}
\usepackage{hyperref}
\usepackage[most]{tcolorbox}
\usepackage{fancyhdr}
\setlength{\parindent}{0.7cm}
\hypersetup{
	colorlinks = true,
	linkcolor = black,
	filecolor = magenta,
	urlcolor = blue,
	}
\urlstyle{same}	
	
\author{
	\\\includegraphics[width=0.7\linewidth]{img/logoPWSZ.eps} \\\\\\\\
	\hfill Karpiński Maciej\\
	\hfill Kuczma Łukasz\\\\
	\hfill Prowadzący mgr inż. Marcin Tracz
	}
\title{\textbf{Zaawansowane metody programowania}\\\line(1,0){400}\\\textbf{laboratorium}}
\date{}

\begin{document}

	%Stron tytułowa
	\maketitle
	\thispagestyle{fancy}
	\fancyhf{}
	\rhead{\textcolor{gray}{\footnotesize Państwowa Wyższa Szkoła Zawodowa im. Witelona w Legnicy\\Informatyka, rok III\\Semestr letni 2020/2021}}	
	\renewcommand{\headrulewidth}{0pt}
	\clearpage

	%Spis treści
	\pagestyle{fancy}
	\rfoot{\thepage}	
	\tableofcontents
	\newpage

	%opis funkcjonalny systemu oraz jego części składowych
	\section{Opis funkcjonalny systemu}	
	\begin{center}
		\includegraphics[width=0.7\linewidth]{img/logo.png}
	\end{center}
	
		
		\indent Projekt ,,Cookbook'' jest systemem do przeglądania oraz zapisywania ulubionych potraw i koktajli. W skład systemu wchodzi:
		\begin{itemize}
			\item Aplikacja desktopowa dla systemu Windows i Linux
			\item Aplikacja webowa
			\item Baza danych Cloud Firestore
			\item System uwierzytelniania Firebase Authentication\\\\
		\end{itemize}
		
		\indent System realizuje następujące funkcjonalności:
		\begin{enumerate}
			\item Aplikacja desktopowa
			\begin{itemize}
				\item Aplikacja dla systemu Windows,
				\item Aplikacja dla systemu Linux,
				\item Logowanie przy pomocy e-maila,
				\item Logowanie przy pomocy Google,
				\item Logowanie przy pomocy Facebooka,
				\item Przeglądanie przepisów na posiłki,
				\item Przeglądanie przepisów na koktajle,
				\item Dodawanie i usuwanie ulubionych przepisów na posiłki,
				\item Dodawanie i usuwanie ulubionych przepisów na koktajle,
				\item Synchronizacja z bazą danych Cloud Firestore.\\\\\\\\\\
			\end{itemize}
					\item Aplikacja webowa
					
			\begin{itemize}
				\item Przeglądanie przepisów posiłków,
							\begin{itemize}
							\item Wyszukiwanie posiłków po nazwie
							\item Wyszukiwanie posiłków po kategorii
							\item Wyszukiwanie posiłków po regionie świata
							\item Wyszukiwanie posiłków po składnikach
							\end{itemize}					
				\item Logowanie przy pomocy e-maila,
				\item Logowania przy pomocy konta Google,
				\item Logowanie przy pomocy konta Facebook,
				\item Dodawanie i usuwanie przepisów do/z ulubionych,

				\item Synchronizacja między aplikacjami,
				\item Obsługa zewnętrznych API z przepisami posiłków,
				\item Obsługa bazy danych Cloud Firestore
			\end{itemize}					
							

		\end{enumerate}
		
		\indent Repozytorium projektu znajduje sie pod adresem: \url{https://github.com/MacKarp/Cookbook} .
	\newpage
	
	%Streszczenie opisu technologicznego każdej aplikacji
	\section{Opis technologiczny}
		\subsection{Aplikacja desktopowa}
			
			\indent Aplikacja desktopowa została stworzona przy wykorzystaniu następujących technologii:
				\subsubsection{Cargo}

					\indent Cargo to menadżer pakietów i system kompilowania języka Rust. Cargo pobiera zależności, kompiluje je, tworzy pakiety do dystrybucji i przesyła je do
					\url{crates.io}, rejestru pakietów społeczności Rust. Cargo można rozbudować o dodatkowe możliwości poprzez instalację dodatkowych pakietów np. watch lub clippy. Większość
					użytkowników używa tego narzędzia do zarządzania swoimi projektami. 
				\subsubsection{chrono}

					\indent Pakiet chrono jest biblioteką dat i godzin dla Rust. Chrono ściśle przestrzega normy ISO 8601, domyślnie rozpoznaje strefę czasową,
				 	posiada oddzielne typy, które nie posiadają strefy czasowej. 
				\subsubsection{Cloud Firestore}

					\indent	Cloud Firestore to elastyczna, skalowalna baza danych do tworzenia aplikacji mobilnych, internetowych i serwerowych z Firebase i Google Cloud.
					Podobnie jak Baza danych czasu rzeczywistego Firebase, zapewnia synchronizację danych między aplikacjami klienckimi za pośrednictwem odbiorników w czasie
					rzeczywistym i oferuje obsługę offline dla urządzeń przenośnych i internetowych, dzięki czemu możesz tworzyć responsywne aplikacje, które działają niezależnie
					od opóźnień w sieci lub łączności z Internetem.
				\subsubsection{Firebase}

					\indent Firebase to platforma opracowana przez Google do tworzenia aplikacji mobilnych i internetowych. Platforma Firebase obejmuje 18 produktów podzielonych
					na trzy grupy:
					\begin{itemize}
						\item Develop,
						\item Quality,
						\item Grow,
					\end{itemize}
					Aplikacja desktopowa wykrzystuje Cloud Firestore i Firebase Authentication.
				\subsubsection{Firebase Authentication}

					\indent	Firebase Authentication zapewnia usługę uwierzytelniania dla backendu, łatwe w użyciu pakiety SDK i gotowe biblioteki interfejsu użytkownika
					do uwierzytelniania użytkowników w aplikacjach. Obsługuje uwierzytelnianie za pomocą haseł, numerów telefonów, popularnych dostawców tożsamości
					federacyjnych, takich jak Google, Facebook i Twitter. Firebase Authentication ściśle integruje się z innymi usługami Firebase i wykorzystuje standardy
					branżowe, takie jak OAuth 2.0 i OpenID Connect.								
				\subsubsection{firestore-db-and-auth}

					\indent Pakiet umożliwiający łatwy dostęp do bazy danych Cloud Firestore za pośrednictwem konta usługi lub poświadczeń OAuth Firebase Authentication.				
				\subsubsection{gdk-pixbuf}

					\indent Pakiet umożliwia wykorzystanie biblioteki Gdk-Pixbuf napisanej w C dla języka Rust. Część projektu gtk-rs.
				\subsubsection{gio}

					\indent Pakiet umożliwia wykorzystanie biblioteki GIO napisanej w C dla języka Rust. Część projektu gtk-rs.
				\subsubsection{Glade}

					\indent Glade jest aplikacją do wizualnego projektowania graficznego interfejsu użytkownika dla programów GTK+/GNOME.
					Projektowany interfejs jest zapisywany jako plik XML. Pliki w formacie GtkBuider i Libglade mogą być ładowane przez odpowiednie biblioteki GTK+ lub Libglade. 
				\subsubsection{glib}

					\indent  Pakiet umożliwia wykorzystanie biblioteki GLib napisanej w C dla języka Rust. Część projektu gtk-rs.
				\subsubsection{gtk}

					\indent Pakiet umożliwia wykorzystanie bibliotek GDK 3, GTK+ 3 i Cairo napisanych w C dla języka Rust. Część projektu gtk-rs.
				\subsubsection{GTK+ 3}

					\indent  Biblioteka służąca do tworzenia interfejsu graficznego do programów komputerowych. Pierwotnie stworzona na potrzeby programu GIMP,
					stąd też nazwa, pochodząca od ang. The GIMP Toolkit. Znak + pojawił się w nazwie, gdy autorzy dodali do oryginalnego GTK możliwość programowania obiektowego.
					GTK+ została napisana w C, aczkolwiek jest zaprojektowana obiektowo, w oparciu o implementację obiektowości dla C – GObject. Z biblioteki GTK+ można korzystać
					przy pomocy większości języków programowania. Biblioteka ta jest podstawą dla środowisk graficznych GNOME i Xfce. Na platformie uniksowej sama wykorzystuje
					bibliotekę GDK (odpowiedzialną za rysowanie obiektów) oraz GLib, zawierającą specjalne typy danych. Dzięki takiemu odseparowaniu GTK+ od systemu graficznego
					(w przypadku Uniksa jest to przeważnie X Window System) biblioteką bezpośrednio odpowiedzialną za interakcję z systemem graficznym, możliwe było łatwe
					przeportowanie GTK+ na inne niż uniksowe architektury (np.: Microsoft Windows oraz linuksowy DirectFB).
				\subsubsection{oauth2}

					\indent Pakiet implementujący protokół OAuth2 (RFC 6749).
				
				\subsubsection{open}
					
					\indent Pakiet open umożliwia otwarcie adresu strony www w domyślnej lub wybranej przeglądarce internetowej.
				\subsubsection{rand}

					\indent Biblioteka Rust przeznaczona do generowania liczb losowych;
				\subsubsection{reqwest}

					\indent Pakiet reqwest to ergonomiczny klient HTTP/HTTPS dla języka Rust.
				\subsubsection{Rust}

					\indent Kompilowany język programowania ogólnego przeznaczenia rozwijany przez Fundację Mozilla. Stworzony z myślą, aby był ,,bezpieczny, współbieżny i praktyczny''.
					Język zaprojektował Graydon Hoare w 2006 roku, w 2009 projekt został przyjęty pod skrzydła Mozilla Foundation. W 2010 Mozilla upubliczniła informację o języku.
					W 2011 roku kompilator języka, znany jako rustc, został z powodzeniem skompilowany przez samego siebie. Pierwsza numerowana wersja alfa została wydana w 2012 roku.
					15 maja 2015 ukazała się wersja 1.0. Rust wykorzystuje Cargo jako menadżer pakietów. Wiele organizacji wykorzystuje ten język programowania w zastosowaniach
					produkcyjnych. Obecnie dwoma największymi otwartymi projektami korzystającymi z języka Rust są: Servo oraz kompilator Rusta. 
				\subsubsection{serde}

					\indent Serde to framework do wydajnej i ogólnej serializacji i deserializacji struktur danych Rust.
				\subsubsection{serde\_ json}

					\indent Serde\_ json jest rozszerzeniem frameworka ,,serde'' o możliwość serializacji i deserializacji struktur danych do i z formatu JSON.
				\subsubsection{url}

					\indent Pakiet wprowadzający typ URL oparty na standardzie URL WHATWG dla języka Rust.
				\subsubsection{Visual studio code}

					\indent Visual Studio Code – darmowy edytor kodu źródłowego z kolorowaniem składni dla wielu języków, stworzony przez Microsoft,
					o otwartym kodzie źródłowym. Oprogramowanie ma wsparcie dla debugowania kodu, zarządzania wersjami kodu źródłowego za pośrednictwem systemu kontroli wersji Git,
					automatycznego uzupełniania kodu IntelliSense, zarządzania wycinkami kodu oraz jego refaktoryzacji. Funkcjonalność aplikacji można rozbudować za pomocą rozszerzeń
					instalowanych z dedykowanego repozytorium rozszerzeń. Według badania przeprowadzonego przez serwis StackOverflow w 2018 roku, Visual Studio Code zostało
					ogłoszone najpopularniejszym narzędziem służącym wytwarzaniu oprogramowania, za którym na drugim miejscu znajduje się produkt tego samego twórcy,
					Microsoft Visual Studio. Oprogramowanie zostało stworzone w oparciu o framework Electron. 
				
\subsection{Aplikacja webowa}

		\subsubsection{HTML}
		\indent HTML to język wykorzystywany do tworzenia i prezentowania stron internetowych www. Jest rozwinięciem języka HTML 4 i jego XML-owej odmiany (XHTML 1), opracowywane w ramach prac grupy roboczej WHATWG (Web Hypertext Application Technology Working Group) i W3C.
		\subsubsection{CSS}
		Kaskadowe arkusze stylów (ang. Cascading Style Sheets, w skrócie CSS) – język służący do opisu formy prezentacji (wyświetlania) stron WWW. CSS został opracowany przez organizację W3C w 1996 r. jako potomek języka DSSSL przeznaczony do używania w połączeniu z SGML-em.
		CSS został stworzony w celu odseparowania struktury dokumentu od formy jego prezentacji. Separacja ta zwiększa zakres dostępności witryny, zmniejsza zawiłość dokumentu, ułatwia wprowadzanie zmian w strukturze dokumentu. CSS ułatwia także zmiany w renderowaniu strony w zależności od obsługiwanego medium (ekran, palmtop, dokument w druku, czytnik ekranowy). Stosowanie zewnętrznych arkuszy CSS daje możliwość zmiany wyglądu wielu stron naraz bez ingerowania w sam kod (X)HTML, ponieważ arkusze mogą być wspólne dla wielu dokumentów.
		\subsubsection{JavaScript}
		JavaScript, w skrócie JS – skryptowy język programowania, stworzony przez firmę Netscape, najczęściej stosowany na stronach internetowych. Twórcą JavaScriptu jest Brendan Eich. W połowie lat 90. XX wieku organizacja ECMA wydała na podstawie JavaScriptu standard języka skryptowego o nazwie ECMAScript, aktualnie rozwijaniem tego standardu zajmuje się komisja TC39.
		\subsubsection{Vue.js}
		Vue.js to front-endowy framework JavaScript typu open source model-view-viewmodel do tworzenia interfejsów użytkownika i aplikacji jednostronicowych. Został stworzony przez Evana You i jest utrzymywany przez niego i pozostałych aktywnych członków zespołu.
		Vue.js zawiera stopniowo adaptowalną architekturę, która koncentruje się na renderowaniu deklaratywnym i komponowaniu komponentów. Podstawowa biblioteka koncentruje się tylko na warstwie widoku. Zaawansowane funkcje wymagane w przypadku złożonych aplikacji, takich jak routing, zarządzanie stanem i narzędzia do budowania, są oferowane za pośrednictwem oficjalnie utrzymywanych bibliotek pomocniczych i pakietów.

Vue.js pozwala rozszerzyć HTML o atrybuty HTML zwane dyrektywami. Dyrektywy oferują funkcjonalność aplikacjom HTML i są dostępne jako dyrektywy wbudowane lub zdefiniowane przez użytkownika. 
		\subsubsection{Cypress}
		Cypress to narzędzie do testowania front-end nowej generacji stworzone z myślą o nowoczesnym internecie. Zajmuje się kluczowymi problemami, z którymi borykają się programiści i inżynierowie QA podczas testowania nowoczesnych aplikacji.
\newline
\newline
Umożliwia:
\begin{itemize}
\item Skonfigurowanie testów
\item Napisanie testów
\item Uruchomienie testów
\item Wykonanie testów debugowania
\end{itemize}

		\subsubsection{Cloud Firestore}

					\indent	Cloud Firestore to elastyczna, skalowalna baza danych do tworzenia aplikacji mobilnych, internetowych i serwerowych z Firebase i Google Cloud.
					Podobnie jak Baza danych czasu rzeczywistego Firebase, zapewnia synchronizację danych między aplikacjami klienckimi za pośrednictwem odbiorników w czasie
					rzeczywistym i oferuje obsługę offline dla urządzeń przenośnych i internetowych, dzięki czemu możesz tworzyć responsywne aplikacje, które działają niezależnie
					od opóźnień w sieci lub łączności z Internetem.\\
					\begin{center}
					\includegraphics[width=1.0 \linewidth]{img/firebase1.jpg}\\
						Widok bazy danych
					\end{center}
		\subsubsection{Firebase}

					\indent Firebase to platforma opracowana przez Google do tworzenia aplikacji mobilnych i internetowych. Platforma Firebase obejmuje 18 produktów podzielonych
					na trzy grupy:
					\begin{itemize}
						\item Develop,
						\item Quality,
						\item Grow,
					\end{itemize}
					Aplikacja desktopowa wykrzystuje Cloud Firestore i Firebase Authentication.
		\subsubsection{Firebase Authentication}

					\indent	Firebase Authentication zapewnia usługę uwierzytelniania dla backendu, łatwe w użyciu pakiety SDK i gotowe biblioteki interfejsu użytkownika
					do uwierzytelniania użytkowników w aplikacjach. Obsługuje uwierzytelnianie za pomocą haseł, numerów telefonów, popularnych dostawców tożsamości
					federacyjnych, takich jak Google, Facebook i Twitter. Firebase Authentication ściśle integruje się z innymi usługami Firebase i wykorzystuje standardy
					branżowe, takie jak OAuth 2.0 i OpenID Connect.								
					
		\begin{center}
			\includegraphics[width=1.0 \linewidth]{img/firebase.jpg}\\
			Zarządzanie kontami użytkowników
		\end{center}	
					
										
					
		\subsubsection{firestore-db-and-auth}

					\indent Pakiet umożliwiający łatwy dostęp do bazy danych Cloud Firestore za pośrednictwem konta usługi lub poświadczeń OAuth Firebase Authentication.
					\subsubsection{Visual studio code}

					\indent Visual Studio Code – darmowy edytor kodu źródłowego z kolorowaniem składni dla wielu języków, stworzony przez Microsoft,
					o otwartym kodzie źródłowym. Oprogramowanie ma wsparcie dla debugowania kodu, zarządzania wersjami kodu źródłowego za pośrednictwem systemu kontroli wersji Git,
					automatycznego uzupełniania kodu IntelliSense, zarządzania wycinkami kodu oraz jego refaktoryzacji. Funkcjonalność aplikacji można rozbudować za pomocą rozszerzeń
					instalowanych z dedykowanego repozytorium rozszerzeń. Według badania przeprowadzonego przez serwis StackOverflow w 2018 roku, Visual Studio Code zostało
					ogłoszone najpopularniejszym narzędziem służącym wytwarzaniu oprogramowania, za którym na drugim miejscu znajduje się produkt tego samego twórcy,
					Microsoft Visual Studio. Oprogramowanie zostało stworzone w oparciu o framework Electron. 
					\subsubsection{Postman}
		Postman to platforma do współpracy podczas tworzenia lub korzystania z API. Funkcje Postmana upraszczają każdy etap tworzenia interfejsu API i usprawniają współpracę, dzięki czemu możesz tworzyć lepsze interfejsy API szybciej.
					\subsubsection{Figma}		
				\indent	Figma to edytor grafiki wektorowej i narzędzie do prototypowania, które jest głównie oparte na sieci Web, z dodatkowymi funkcjami offline dostępnymi w aplikacjach komputerowych dla systemów macOS i Windows.
		\begin{center}
		\includegraphics[width=1.0 \linewidth]{img/figma.jpg}\\
		Projekt strony ze szczegółami posiłku, stworzony w Figma
	\end{center}		
	\newpage
	
	%Strzeszczenie wykorzystywanych wzorców projektowych w każdej aplikacji
	\section{Wzorce projektowe}
		\subsection{Aplikacja desktopowa}

			\indent Pakiet bibliotek gtk-rs wykorzystuje następujące wzorce projektowe:
			\begin{itemize}
				\item Adapter
				\item Builder
				\item Singleton								 
			\end{itemize}

			\indent Pakiet firestore-db-and-auth wykorzystuje wzorce projektowe:
			\begin{itemize}
				\item Data mapper
				\item Repository
			\end{itemize}
			Główny obiekt przechowujący informacje dotyczącą interfejsu ,,GuiData'' wykorzystuje wzorce projektowe:
			\begin{itemize}
				\item Builder
				\item Pool
				\item Prototype
			\end{itemize}

			\indent Kod obsługujący API \href{https://www.thecocktaildb.com/api.php}{TheCocktailDB} i \href{https://www.themealdb.com/api.php}{TheMealDB}
			wykorzystuje następujące wzorce projektowe:
			\begin{itemize}
				\item Data mapper
				\item Chain of Responsibilities
				\item Repository
			\end{itemize}

			\indent Gdzie tylko było to możliwe wykorzystane w kodzie zostały następujące wzorce projektowe:
			\begin{itemize}
			\item Iterator
			\item Fluent Interface
			\item Dependency injection
			\end{itemize}			 

		\subsection{Aplikacja webowa}	
		Podczas tworzenia aplikacji webowej zastosowane zostały następujące wzorce projektowe:
		\begin{itemize}
			\item Polecenie (Przycisk/komponent dodawania posiłku do ulubionych)
			\item Iterator (Wykorzystywany podczas odczytu kolekcji)
			\item Metoda szablonowa (Komponenty Vue)
			\end{itemize} 
	\newpage
	
	%Instrukcja lokalnego i zdalengo uruchamiania testów oraz samego systemu
	\section{Instrukcja uruchamiania testów i systemu}
		\subsection{Aplikacja desktopowa}

			\indent Testy można uruchomić jedynie w systemie operacyjnym Linux. W celu lokalnego uruchomienia testów aplikacji należy w katalogu ,,Desktop''
			wywołać w terminalu polecenie:
			\begin{tcolorbox}[minipage,colback=white,arc=0pt,outer arc=0pt, fontupper=\normalsize]
				\center					
					cargo test -- --test-threads=1
			\end{tcolorbox}

			\indent Zdalne uruchomienie testów odbywa się automatycznie poprzez platformę \href{https://www.travis-ci.com/github/MacKarp/Cookbook}{Travis CI} 
			przy każdym ,,pushu'' do repozytorium GitHub
			 
			\indent Po pobraniu archiwum przeznaczonego dla systemu Windows lub Linux z strony \href{https://github.com/MacKarp/Cookbook/releases}{GitHub Releases},
			należy je wypakować a następnie uruchomić przy wykorzystaniu pliku wykonywalnego o nazwie gui.exe(Windows) lub aktywatora(Linux) o nazwie ,,Cookbook''. 		
		\subsection{Aplikacja webowa}	 
		\indent Aplikacje uruchomić można pod adresem: \href{https://cookbook-307109.web.app}{https://cookbook-307109.web.app} \\\\
	\indent Testy integracyjne realizowane są za pomocą narzędzia Cypress. Środowiskiem testowym jest środowisko developerskie. Aby uruchomić testy należ odpalić serwer lokalny poprzez terminal (npm run serve). Po uruchomieniu serwera należy włączyć narzędzie testujące komendą \textit{npm run cypress:open}. Otwarte zostanie okno aplikacji Cypress.\\
		\newline
	\includegraphics[width=1.0\linewidth]{img/cypress.jpg}\\\\
	\newline
	Do dyspozycji są tu dwa pliki zawierające szereg testów.
	\newline
	\newline
	\includegraphics[width=1.0\linewidth]{img/cypres1.jpg}\\
	\newline
	$Plik favorite_test.js$
	
	\newpage

	%wnioski projektowe
	\section{Wnioski projektowe}
		\subsection{Aplikacja desktopowa}
	
		\indent Tworzenie aplikacji desktopowej przy użyciu języka programowania Rust było mniejszym wyzwaniem niż początkowo się wydawało, praca nad projektem bardzo
		rozwinęła moją znajomość tego języka programowania. Brak oficjalnego SDK do API Google Firebase i słabo rozwinięte biblioteki stworzone przez użytkowników
		sprawiały niekiedy problemy przez co musiałem część funkcjonalności stworzyć samemu. Wybór technologii GTK do stworzenia interfejsu użytkownika był nowym i interesującym
		doświadczeniem, jedynymi problemami okazała się słaba dokumentacja dotycząca kompilacji aplikacji dla systemu Windows, oraz niekompatybilność biblioteki webkit
		(o czym w czasie wyboru technologii nie wiedziałem) z systemem Windows przez co jedna z zaplanowanych funkcjonaliści nie działa do końca poprawnie. 
		Tworząc aplikacje nauczyłem się wielu rzeczy takich jak cross kompilacja, autoryzacja OAuth2, obsługa Google Firebase, Travis CI czy wykorzystanie zewnętrznego API.

		\subsection{Aplikacja webowa}	 
\indent Mimo początkowych problemów, praca z frameworkiem Vue okazała się dla mnie dość przejrzysta i przyjemna. Możliwość tworzenia komponentów i wykorzystywania ich ponownie w różnych obszarach aplikacji jest niezwykle przyjemna i oszczędza czas przeznaczony na ponowne pisanie kodu. Możliwości manipulacji elementami DOM w html, oraz zastosowania narzędzi programistycznych takich jak pętle i warunki plus JavaScript, dają niemal nieskończone możliwości w tworzeniu aplikacji webowych. Jestem pewien, że wykorzystam to narzędzie podczas tworzenia aplikacji będącej częścią mojej pracy dyplomowej. \\\\
Dodatkowo pozytywnym zaskoczeniem jest dla mnie platforma Firebase. Oferuje mnóstwo możliwości od tworzenia baz danych przez udostępnianie miejsca na zasoby po hosting. 
\end{document}
