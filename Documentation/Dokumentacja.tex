%ustawienia
\documentclass[12pt,a4paper]{article}
\usepackage[T1]{fontenc}
\usepackage{mathptmx}
\usepackage[utf8]{inputenc}
\usepackage{amssymb}
\usepackage[polish]{babel}
\usepackage{polski}
\usepackage{amsmath}
\usepackage{amsfonts}
\usepackage[left=3.5cm,right=2cm,top=2.5cm,bottom=2.5cm]{geometry}
\usepackage{graphicx}
\usepackage{indentfirst} 
\usepackage{float}
\usepackage{hyperref}
\usepackage[most]{tcolorbox}
\usepackage{fancyhdr}
\setlength{\parindent}{0.7cm}
\hypersetup{
	colorlinks = true,
	linkcolor = black,
	filecolor = magenta,
	urlcolor = blue,
	}
\urlstyle{same}	
	
\author{
	\\\includegraphics[width=0.7\linewidth]{img/logoPWSZ.eps} \\\\\\\\
	\hfill Karpiński Maciej\\
	\hfill Kuczma Łukasz\\
	\hfill Zawada Marcin\\\\
	\hfill Prowadzący mgr inż. Marcin Tracz
	}
\title{\textbf{Zaawansowane metody programowania}\\\line(1,0){400}\\\textbf{laboratorium}}
\date{}

\begin{document}

	%Stron tytułowa
	\maketitle
	\thispagestyle{fancy}
	\fancyhf{}
	\rhead{\textcolor{gray}{\footnotesize Państwowa Wyższa Szkoła Zawodowa im. Witelona w Legnicy\\Informatyka, rok III\\Semestr letni 2020/2021}}	
	\renewcommand{\headrulewidth}{0pt}
	\clearpage

	%Spis treści
	\pagestyle{fancy}
	\rfoot{\thepage}	
	\tableofcontents
	\newpage

	%opis funkcjonalny systemu oraz jego części składowych
	\section{Opis funkcjonalny systemu}	
	\newpage
	
	%Streszczenie opisu technologicznego każdej aplikacji
	\section{Opis technologiczny}
		\subsection{Aplikacja desktopowa}
			
			\indent Aplikacja desktopowa została stworzona przy wykorzystaniu następujących technologii:
				\subsubsection{Cargo}

					\indent Cargo to menadżer pakietów i system kompilowania języka Rust. Cargo pobiera zależności, kompiluje je, tworzy pakiety do dystrybucji i przesyła je do
					\url{crates.io}, rejestru pakietów społeczności Rust. Cargo można rozbudować o dodatkowe możliwości poprzez instalację dodatkowych pakietów np. watch lub clippy. Większość
					użytkowników używa tego narzędzia do zarządzania swoimi projektami. 
				\subsubsection{chrono}

					\indent Pakiet chrono jest biblioteką dat i godzin dla Rust. Chrono ściśle przestrzega normy ISO 8601, domyślnie rozpoznaje strefę czasową,
				 	posiada oddzielne typy, które nie posiadają strefy czasowej. 
				\subsubsection{Cloud Firestore}

					\indent	Cloud Firestore to elastyczna, skalowalna baza danych do tworzenia aplikacji mobilnych, internetowych i serwerowych z Firebase i Google Cloud.
					Podobnie jak Baza danych czasu rzeczywistego Firebase, zapewnia synchronizację danych między aplikacjami klienckimi za pośrednictwem odbiorników w czasie
					rzeczywistym i oferuje obsługę offline dla urządzeń przenośnych i internetowych, dzięki czemu możesz tworzyć responsywne aplikacje, które działają niezależnie
					od opóźnień w sieci lub łączności z Internetem.
				\subsubsection{Firebase}

					\indent Firebase to platforma opracowana przez Google do tworzenia aplikacji mobilnych i internetowych. Platforma Firebase obejmuje 18 produktów podzielonych
					na trzy grupy:
					\begin{itemize}
						\item Develop,
						\item Quality,
						\item Grow,
					\end{itemize}
					Aplikacja desktopowa wykrzystuje Cloud Firestore i Firebase Authentication.
				\subsubsection{Firebase Authentication}

					\indent	Firebase Authentication zapewnia usługę uwierzytelniania dla backendu, łatwe w użyciu pakiety SDK i gotowe biblioteki interfejsu użytkownika
					do uwierzytelniania użytkowników w aplikacjach. Obsługuje uwierzytelnianie za pomocą haseł, numerów telefonów, popularnych dostawców tożsamości
					federacyjnych, takich jak Google, Facebook i Twitter. Firebase Authentication ściśle integruje się z innymi usługami Firebase i wykorzystuje standardy
					branżowe, takie jak OAuth 2.0 i OpenID Connect.								
				\subsubsection{firestore-db-and-auth}

					\indent Pakiet umożliwiający łatwy dostęp do bazy danych Cloud Firestore za pośrednictwem konta usługi lub poświadczeń OAuth Firebase Authentication.				
				\subsubsection{gdk-pixbuf}

					\indent Pakiet umożliwia wykorzystanie biblioteki Gdk-Pixbuf napisanej w C dla języka Rust.
				\subsubsection{gio}

					\indent Pakiet umożliwia wykorzystanie biblioteki GIO napisanej w C dla języka Rust.
				\subsubsection{Glade}

					\indent Glade jest aplikacją do wizualnego projektowania graficznego interfejsu użytkownika dla programów GTK+/GNOME.
					Projektowany interfejs jest zapisywany jako plik XML. Pliki w formacie GtkBuider i Libglade mogą być ładowane przez odpowiednie biblioteki GTK+ lub Libglade. 
				\subsubsection{glib}

					\indent  Pakiet umożliwia wykorzystanie biblioteki GLib napisanej w C dla języka Rust.
				\subsubsection{gtk}

					\indent Pakiet umożliwia wykorzystanie bibliotek GDK 3, GTK+ 3 i Cairo napisanych w C dla języka Rust.
				\subsubsection{GTK+ 3}

					\indent  Biblioteka służąca do tworzenia interfejsu graficznego do programów komputerowych. Pierwotnie stworzona na potrzeby programu GIMP,
					stąd też nazwa, pochodząca od ang. The GIMP Toolkit. Znak + pojawił się w nazwie, gdy autorzy dodali do oryginalnego GTK możliwość programowania obiektowego.
					GTK+ została napisana w C, aczkolwiek jest zaprojektowana obiektowo, w oparciu o implementację obiektowości dla C – GObject. Z biblioteki GTK+ można korzystać
					przy pomocy większości języków programowania. Biblioteka ta jest podstawą dla środowisk graficznych GNOME i Xfce. Na platformie uniksowej sama wykorzystuje
					bibliotekę GDK (odpowiedzialną za rysowanie obiektów) oraz GLib, zawierającą specjalne typy danych. Dzięki takiemu odseparowaniu GTK+ od systemu graficznego
					(w przypadku Uniksa jest to przeważnie X Window System) biblioteką bezpośrednio odpowiedzialną za interakcję z systemem graficznym, możliwe było łatwe
					przeportowanie GTK+ na inne niż uniksowe architektury (np.: Microsoft Windows oraz linuksowy DirectFB).
				\subsubsection{oauth2}

					\indent Pakiet implementujący protokół OAuth2 (RFC 6749).
				\subsubsection{rand}

					\indent Biblioteka Rust przeznaczona do generowania liczb losowych;
				\subsubsection{reqwest}

					\indent Pakiet reqwest ergonomiczny klient HTTP/HTTPS dla języka Rust.
				\subsubsection{Rust}

					\indent Kompilowany język programowania ogólnego przeznaczenia rozwijany przez Fundację Mozilla. Stworzony z myślą, aby był ,,bezpieczny, współbieżny i praktyczny''.
					Język zaprojektował Graydon Hoare w 2006 roku, w 2009 projekt został przyjęty pod skrzydła Mozilla Foundation. W 2010 Mozilla upubliczniła informację o języku.
					W 2011 roku kompilator języka, znany jako rustc, został z powodzeniem skompilowany przez samego siebie. Pierwsza numerowana wersja alfa została wydana w 2012 roku.
					15 maja 2015 ukazała się wersja 1.0. Rust wykorzystuje Cargo jako menadżer pakietów. Wiele organizacji wykorzystuje ten język programowania w zastosowaniach
					produkcyjnych. Obecnie dwoma największymi otwartymi projektami korzystającymi z języka Rust są: Servo oraz kompilator Rusta. 
				\subsubsection{serde}

					\indent Serde to framework do wydajnej i ogólnej serializacji i deserializacji struktur danych Rust.
				\subsubsection{serde\_ json}

					\indent Serde\_ json jest rozszerzeniem frameworka o możliwość serializacji i deserializacji struktur danych do/z formatu JSON.
				\subsubsection{url}

					\indent Biblioteka URL dla Rust, oparta na standardzie URL WHATWG.
				\subsubsection{Visual studio code}

					\indent Visual Studio Code – darmowy edytor kodu źródłowego z kolorowaniem składni dla wielu języków, stworzony przez Microsoft,
					o otwartym kodzie źródłowym. Oprogramowanie ma wsparcie dla debugowania kodu, zarządzania wersjami kodu źródłowego za pośrednictwem systemu kontroli wersji Git,
					automatycznego uzupełniania kodu IntelliSense, zarządzania wycinkami kodu oraz jego refaktoryzacji. Funkcjonalność aplikacji można rozbudować za pomocą rozszerzeń
					instalowanych z dedykowanego repozytorium rozszerzeń. Według badania przeprowadzonego przez serwis StackOverflow w 2018 roku, Visual Studio Code zostało
					ogłoszone najpopularniejszym narzędziem służącym wytwarzaniu oprogramowania, za którym na drugim miejscu znajduje się produkt tego samego twórcy,
					Microsoft Visual Studio. Oprogramowanie zostało stworzone w oparciu o framework Electron. 
				\subsubsection{webkit2gtk}

					\indent Pakiet umożliwia wykorzystanie biblioteki webkit2gtk napisanej w C dla języka Rust.

		\subsection{Aplikacja mobilna}
		\subsection{Aplikacja webowa}
	\newpage
	
	%Strzeszczenie wykorzystywanych wzorców projektowych w każdej aplikacji
	\section{Wzorce projektowe}
		\subsection{Aplikacja desktopowa}

			\indent Pakiet bibliotek gtk-rs wykorzystuje następujące wzorce projektowe:
			\begin{itemize}
				\item Adapter
				\item Builder
				\item Singleton								 
			\end{itemize}

			\indent Biblioteka firestore-db-and-auth wykorzystuje wzorce projektowe:
			\begin{itemize}
				\item Data mapper
				\item Repository
			\end{itemize}
			Główny obiekt przechowujący informacje dotyczącą interfejsu ,,GuiData'' wykorzystuje wzorce projektowe:
			\begin{itemize}
				\item Builder
				\item Pool
				\item Prototype
			\end{itemize}

			\indent Kod obsługujący API \href{https://www.thecocktaildb.com/api.php}{TheCocktailDB} i \href{https://www.themealdb.com/api.php}{TheMealDB}
			wykorzystuje następujące wzorce projektowe:
			\begin{itemize}
				\item Data mapper
				\item Chain of Responsibilities
				\item Repository
			\end{itemize}

			\indent Gdzie tylko było to możliwe wykorzystany w kodzie zostały następujące wzorce projektowe:
			\begin{itemize}
			\item Iterator
			\item Płynny interfejs
			\item Wstrzykiwanie zależności
			\end{itemize}			 
		\subsection{Aplikacja mobilna}
		\subsection{Aplikacja webowa}	 
	\newpage
	
	%Instrukcja lokalnego i zdalengo uruchamiania testów oraz samego systemu
	\section{Instrukcja uruchamiania testów i systemu}
		\subsection{Aplikacja desktopowa}

			\indent Testy można uruchomić jedynie w systemie operacyjnym Linux. W celu lokalnego uruchomienia testów aplikacji należy w katalogu ,,Desktop''
			uruchomić w terminalu polecenie:
			\begin{tcolorbox}[minipage,colback=white,arc=0pt,outer arc=0pt, fontupper=\normalsize]
				\center					
					cargo test -- --test-threads=1
			\end{tcolorbox}

			\indent W celu zdalnego uruchomienia testów należy udać się na stronę \href{https://github.com/MacKarp/Cookbook/actions/workflows/DesktopCI.yml}{GitHub Actions}.
			 
			\indent Po pobraniu archiwum przeznaczonego dla systemu Windows lub Linux z strony \href{https://github.com/MacKarp/Cookbook/releases}{GitHub Releases},
			należy je wypakować a następnie uruchomić przy wykorzystaniu skrótu(Windows) lub aktywatora(Linux) o nazwie ,,Cookbook''. 		
		\subsection{Aplikacja mobilna}
		\subsection{Aplikacja webowa}	 
	
	\newpage

	%wnioski projektowe
	\section{Wnioski projektowe}
	\newpage

	\newpage	
	\section{Spis rysunków}
		\listoffigures
\end{document}
